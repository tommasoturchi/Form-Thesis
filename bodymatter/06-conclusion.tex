% !TEX root = ../thesis.tex
%
\setcounter{chapter}{5}
\chapter{Conclusion}\label{chap:conclusion}

\cleanchapterquote{Science is a way of thinking much more than it is a body of knowledge.}{Carl Sagan}{Science Friday interview, 1996}

This chapter concludes the thesis by discussing the results of all the studies carried out in relation to each investigated Research Question and in light of the main one set out to be addressed in the introduction. The research contributions and implications are laid out, highlighting also its current limitations. Finally, possible future research directions are summarised.

\section{Summary}
Thanks to the amount of software driving our whole lives, coding and \ac{CL} have become essential skills for an ever wider audience. Nonetheless, the highly abstract concepts involved by them still constitute a huge barrier to a widespread appropriation of technology for the general public. Yet, these skills are not just related to programming itself, but they contribute to developing the so-called \acl{CT}, namely all those thinking abilities reflecting core concepts\added[comment={D64}]{ and practices} of \ac{CS}. This set of skills can enable people to actively participate and address today's challenges with the help of modern technological tools, solving complex problems and express their solutions using a computer.

This thesis set out to investigate the effects of physical manipulation on the development of \ac{CT} skills: according to the constructivist theories of Jean Piaget~\cite{Piaget:1969vq}, exploiting human's innate dexterity for objects manipulation in the physical world and its afforded social interactions could be an effective way of aiding users in practising abstract concepts as \ac{CT}. Physical manipulation sits at the core of \acp{TUI}, a digital interaction paradigm designed with the aim of providing users with an easy to use interface that can benefit inexperienced people. Such a paradigm, often used to support the interaction of young children with technology in the classroom, could be employed to promote \ac{CT} skills by providing users with a physical representation of the concepts involved~\cite{McNerney:2004jc,Horn:2009be}, acting as a scaffold between the real world and digital~\cite{Wang:2014jy}.

Chapter~\ref{chap:exploration} presented an overview of existing tools based on \acp{VPL} currently used in programming sessions around the world to introduce \ac{CT} to the most diverse audiences. \replaced[comment={MP3, MP4, D65, D66, D74}]{An investigation was carried out to verify to what extent --- if at all --- such tools support the collaborative learning of \ac{CT} skills, specifically of two computational Concepts~\cite{Brennan:2012}, sequences and loops}{The investigation was related to how such tools support the collaborative learning of \ac{CT} skills, in order to investigate whether they might be overlooking it}. The reported quantitative study compared collaboratively working learners with individual working ones for the purpose of evaluating \ac{CT} skills development in the context of using a \ac{VPL}-based technological tool in a real-world introductory programming session. Unfortunately, the results of the study didn't show any significance, but the lessons learned in designing and carrying out the study were used in the following experiments to design better tools that support \ac{CT} skills in collaborative environments.

Chapter~\ref{chap:tapas} addressed the main thesis investigation over the effects of physical manipulation on the development of \ac{CT} skills in a specific set of educational domains, namely \ac{IL} ones, where learning is mostly self-directed and takes place as people go about their daily activities, driven by their preferences and intentions. A preliminary design of \acr{TAPAS} was presented, a software platform combining its digital and physical features to promote \ac{CT} skills in different \ac{IL} domains. \ac{TAPAS}' design stems from a workshop with expert designers used to collect insightful ideas and design challenges related to its development. \ac{TAPAS} was then used to investigate the effects of physical manipulation on the development of such skills through a two-phase qualitative study carried out both with undergraduates working in groups and expert designers. The results showed that \ac{TAPAS} provides a positive user experience and could be used effectively in \ac{IL} scenarios; a potential side effect caused by employing it to support learning might be a development of \replaced[comment={MP3, D66, D74}]{those \ac{CT} skills associated with the computational Concept of sequences, and the practice of abstracting and modularizing~\cite{Brennan:2012}}{\ac{CT} skills} thanks to its design rationale, but more studies are needed in order to investigate this effect further.

Finally, Chapter~\ref{chap:tapasplay} took an extra step towards addressing the main thesis Research Question and presented an investigation on the effects of physical manipulation on the development of \ac{CT} skills through gameplay activities, a common scenario often used in introductory programming courses. An extension of \ac{TAPAS}, called TAPASPlay, was presented to address it, which consists of a turn-taking serious game using gameplay to foster \ac{CT} skills by making learners experience engaging and social. The developed prototype was employed in a study with a group of secondary school girls, whose results showed some evidence that TAPASPlay might offer an engaging and playful environment to develop \ac{CT} skills\added[comment={MP3, D66, D74}]{, specifically in relation to the computational Concept of sequences, and the Practices of being incremental and iterative, and abstracting and modularizing~\cite{Brennan:2012}}.

\section{Research Contributions}
The main research contributions of this work concern the \replaced[comment={D67}]{use}{development} of \ac{TAPAS} as a repurposable tool to study the effects of physical manipulation on the development of \ac{CT}. It was designed following the results of a workshop with experts that shaped it to be repurposed to different \ac{IL} scenarios. \ac{TAPAS} was then evaluated in a sample scenario and showed promising results in terms of provided user experience; it was then repurposed (as TAPASPlay) to a different \ac{IL} scenario through gameplay, showing promising results in relation to \ac{CT} skills development and user engagement.

The feasibility of repurposing \ac{TAPAS} to different \ac{IL} domains have been \replaced[comment={D68}]{shown, allowing it to be used}{demonstrated, prompting for its use} in many other scenarios to support \replaced[comment={MP4, D74}]{learning sequences and abstracting and modularizing}{the learning of abstract concepts}. TAPASPlay demonstrated that supporting \ac{CT} with gameplay and physical manipulation can \replaced[comment={D69}]{prompt for a sustained user engagement while offering}{result in high user engagement, increased collaboration, and offers} new ways of assisting skill progression amongst learners\added[comment={MP4, D74}]{, with respect to computational practices like abstracting and modularizing and being incremental and iterative}.

Moreover, the results of the study reported in Chapter~\ref{chap:exploration} highlighted the need for developing new \replaced[comment={MP4, D74}]{\ac{CT} tools supporting learning sequences and loops}{tools supporting \ac{CT} skills} that better leverage on collaboration amongst peers to enhance learning.

Finally, some of the main challenges faced by supporting \ac{CT} skills with physical objects manipulation in \ac{IL} domains were highlighted, based on insights from a two-phase study carried out with end-users and interaction designers.

To recap, the main Research Question addressed by this thesis was: \textit{``Can the collaborative and cognitive naturalness of physical objects manipulation at the basis of \aclp{TUI} aid the understanding of core algorithmic principles and \replaced[comment={D1}]{thus improve}{thus improving} end-users' \acl{CT} skills?''}.

The following Key Research Questions were formulated and addressed throughout the thesis in order to support and investigate the main Research Question in detail:
\begin{itemize}
    \item \textit{``Do existing \ac{VPL}-based tools support the collaborative learning of \ac{CT} skills?''} From the results of the study carried out in Chapter~\ref{chap:exploration}, a definite answer cannot be yet provided, but further studies are needed to show that existing tools are leveraging on collaborative learning.
    \item \textit{``Can physical objects manipulation help foster \acl{CT} skills in \acl{IL} domains?''} The results of the two-phases study carried out in Chapter~\ref{chap:tapas} are promising, suggesting that \acp{TUI} can provide support for developing \ac{CT} skills in such domains.
    \item \textit{``Can physical objects manipulation provide a playful and engaging way of learning \ac{CT} skills through gameplay?''} The results of the study reported in Chapter~\ref{chap:tapasplay} suggest that combining a \ac{TUI} with gameplay can develop \ac{CT} skills and support the collaborative learning.
\end{itemize}

Ultimately, further studies are needed to fully address the main Research Question, but from the preliminary studies carried out and reported in this thesis, one can argue that physical manipulation provides support for developing \ac{CT} skills and might represent the natural evolution of existing tools currently used in educational environments.\added[comment={D71}]{ Future studies should cover more dimensions of \ac{CT} as defined by Brennan and Resnick~\cite{Brennan:2012}, with particular reference to \ac{CT} Perspectives, which are usually hard to capture. Different Design Scenarios should be developed, with a more extensive set of instructions that can be issues by users in order to provide more breadth to the available measures.}

\section{Research Limitations}
The research reported in this thesis has some limitations that highlight the need for further future research.

\paragraph{Scenarios} The two main studies directly investigating the main thesis Research Question in Chapters~\ref{chap:tapas} and~\ref{chap:tapasplay} were carried out in Brunel Facilities with students from either the University itself or the surrounding High Schools. Replication studies are needed in order to generalise their findings to students from other areas and --- since supporting \ac{CT} in \ac{IL} domains relates to a very heterogeneous audience --- to other age groups. Familiarity with technology should be another confounding variable worth considering in these cases.

\paragraph{Perspectives} The \ac{TAPAS} platform was tested only from a user perspective, analysing its effects on supporting \ac{CT} skills. On the other hand, its architecture, as discussed in Section~\ref{sec:tapasarch}, is meant to allow its repurposing to different scenarios; indeed, \ac{TAPAS} was repurposed to a different \ac{IL} domain in combination with gameplay activities and rebranded as TAPASPlay in Chapter~\ref{chap:tapasplay}. This process was carried out by its original author, thus the ease of such activity needs to be properly evaluated by analysing it from a developer perspective. For this reason, \ac{TAPAS}' source code is going to be published with an open source license, allowing other developers to repurpose it do different domains.

\paragraph{Sample Sizes} Most of the studies carried out and reported in this thesis involved small groups of participants, typically less than 20. Such small groups of students were appropriate for those preliminary qualitative studies, but bigger and longitudinal ones are needed to reveal the real effects of using the proposed systems and investigate initial claims.

\paragraph{Significance} The study in Chapter~\ref{chap:exploration} was carried out to evaluate \ac{CT} skills development in the context of using a \ac{VPL}-based tool for collaboratively working learners; even though the sample size was quite substantial (88 students), the results failed to show statistical significance, which prompts for further investigations over this matter.

\paragraph{Assessment} As pointed out a number of times throughout the thesis, the research community still lacks an accepted definition of \ac{CT} and, thus, a unified way of measuring it. In carrying out the studies reported, different (mostly qualitative) measures have been employed to attempt to capture effects correlated with the development of such skills, but if new methods and tools need to be designed to better support learners, researchers must keep on investigating this matter and devise an appropriate framework that can be used, highlighting the pros and cons of existing ones.

\section{Fostering Computational Thinking Skills}
In this thesis, a range of tools and methods supporting \ac{CT} skills have been proposed. Many useful pointers have been raised throughout this work, which are collected and summarised in the following.

Collaboration --- as suggested by Piaget's con\-struc\-tiv\-ist the\-o\-ry of learn\-ing~\cite{Piaget:1969vq} --- pro\-vides a prom\-is\-ing way of fostering \ac{CT} skills in different scenarios, but seems slightly overlooked by current research in this field.

\acp{TUI} present an engaging way of fostering \ac{CT} skills by supporting users in practising abstract concepts by leveraging on physical objects manipulation and encourage collaboration amongst users, which in turns support their learning activities.

Gameplay could be used to engage young girls in \ac{STEM} activities, empowering them with the right tools to actively participate and take control of the issues coming up in the future, whilst allowing them to take on a more central role in the science and technology sector.

\section{Future Work}
\paragraph{Support more tangibles} By taking full advantage of the \acp{TUIReO} protocol discussed in Section~\ref{sec:tapasarch}, \ac{TAPAS} deployments can support a wide range of tangible objects, which might expose their specific function and provide digital features based on their shape, exploiting their physical affordance.

\paragraph{Reduce setup requirements} The setup required to run \ac{TAPAS} is quite complex and requires specific hardware that needs to be mounted in dedicated spaces. Further work should optimize its digital footprint and requirements, in order to ease the needed setup and enable its ubiquitous deployments.