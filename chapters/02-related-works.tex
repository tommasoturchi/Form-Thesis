% !TEX root = ../thesis.tex
%
\chapter{Related Works}\label{sec:related}

\cleanchapterquote{Users do not care about what is inside the box, as long as the box does what they need done.}{Jef Raskin}{about Human Computer Interfaces}

In this chapter a survey of the relevant literature is presented. The chapter offers an overview of the research surrounding Computational Thinking, from how the concept evolved and differentiated itself from Computational Literacy, the many proposed definitions and the many ways it has been evaluated so far, outlining other similar mental skills often associated with it. It then provides a high-level guide over Tangible-User Interfaces and how they 

\section{Computational Thinking} %- What it is generally and why it is relevant - Success stories, where and how
Since the coming of the Computer revolution and Information age, technology have progressively taken an increasingly bigger part in our daily lives; from the simple task of turning on the engine of your car to the vastly more complicated and entwined processes surrounding the management of a flight, it is unmistakably clear how much our society depends on software: technology surrounds every aspect of our lives, thus being computationally literate and knowning one's way into such technologies are becoming much needed skills to possess for an ever wider and etherogeneous audience. The renewed interest in such discipline is witnessed by many public initiatives that have had quite a success in the past few years, and more are coming along and following the same steps: Code.org’s Hour of Code\footnote{\url{test}} is the main example of a successful global initiative involving millions of students of different ages starting with 4-year old and aims at introducing coding to a wide audience with different backgrounds. Many known figures have stressed the importance of possessing such skills: Stephen Wolfram wrote a blog post stating how \TODO a new that went viral, President Obama pledged to provide \$4 billion\footnote{CITE} in funding for computer science education in U.S. schools as part of the Computer Science for All initiative announced in 2016.

This and similar initiatives --- as with coding itself --- are not just about programming though: they endeavor to foster Computational Thinking skills, namely all those thinking abilities lying at the hearth of Computer Science, such as problem solving, abstraction, and pattern recognition.

It is unmistakably clear how software surrounds every aspect of our lives, making computational literacy and coding much needed skills to possess for an ever wider audience. For this very reason, many emerging initiatives such as the Hour of Code have been successful in introducing them to wide audiences with heterogeneous backgrounds. nonetheless, these initiatives --- as with coding itself --- are not just about programming though as they endeavour to foster Computational Thinking skills, namely all those thinking abilities lying at the heart of Computer Science itself, such as problem solving, abstraction, and pattern recognition.

\subsection{Computational Literacy} % - Where CT came from - context - Old definitions and implications

\subsection{Defining Computational Thinking} % - Current definitions - from Wing - and struggle to define it - MIT framework
Wing first defined Computational Thinking (CT) as a set of thinking skills, habits, and approaches that are needed to solve complex problems using a computer and widely applicable in our information society. It covers far more than programming itself, including a range of mental tools reflecting fundamental principles and concepts of Computer Science, such as abstracting and decomposing a problem, identifying recurring patterns and being able to generalize solutions.

\subsection{Measuring Computational Thinking} % - Different measures - entangled with definition

\subsection{Computational vs Design vs Mathematical Thinking} % - What are other thinking skills and what is the difference?

\section{End-User Development}

\subsection{Programming by Instruction vs Demonstration}

\section{Tangible User Interfaces}
Declining hardware costs have recently enabled many new technologies to be available to a wider audience, together with new and engaging interaction modalities, particularly using gestures or object movements; this revolutionary paradigm goes under the name of the \ac{NUI}, and it allows people to act and communicate with digital systems in ways to which they are naturally predisposed.

The term ``natural'' has been used in a rather loose fashion, meaning intuitive, easy to use or easy to learn; many studies argue that we can design a natural interaction either by mimicking aspects of the real world \cite{Jacob:2008dj} or by drawing on our existing capabilities in the communicative or gesticulative areas \cite{Wigdor:2011:BNW:1995309}.

One of the most successful and developed approaches falling into the first category has been introduced by Ishii et al. \cite{Ishii:1997gy} and is known as \acp{TUI}. The aim of \acp{TUI} is to give bits a directly accessible and manipulable interface by employing the real world, both as a medium and as a display for manipulation; indeed by connecting data with physical artifacts and surfaces we can make bits tangible. 

Many studies in this research area investigate the supposed benefits offered by this interaction paradigm, ranging from intuitiveness \cite{Ishii:1997gy}, experiential learning through direct manipulation \cite{Manches:2009kg, Parkes:2008bu}, motor memory \cite{Weiss:2009ct}, accuracy \cite{Muller:2014kx}, and collaboration \cite{Subramanian:2007kx}. Furthermore, the effects of employing a \ac{TUI} to interact with a digital system are certainly dependent on the tasks and domain, as many comparative studies suggest \cite{Weiss:2009ct, Muller:2014kx, Hancock:2009bg}; for this reason, Kirk et al. \cite{Kirk:2009ue} made the case for hybrid surfaces, employing physical elements together with digital ones.

Researchers are also debating how employing \acp{TUI} reflects on learning \cite{Horn:2009be,Marshall:2007dr,Antle:2013ho}, with specific reference to highly abstract concepts: this stems from Piagetian theories supporting the development of thinking --- particularly in young children --- through manipulation of concrete physical objects. Other studies \cite{Wang:2014jy,Horn:2011ch} are even linking this effect to the development of Computational Thinking skills \cite{Wing:2006iz}, namely a new kind of analytical thinking integral to solving complex problems using core computer scientists' tools, such as abstraction and decomposition.

Due to the ubiquitous nature of our scenario and the aforementioned traits of \acp{TUI}, we felt that designing our system around a tangible interaction would contribute to fostering its usage in a more sustained and prolonged way.

\subsection{Natural User Interfaces}

\subsection{Reality-Based Interfaces}

\section{Conclusion}