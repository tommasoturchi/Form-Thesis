% !TEX root = ../thesis.tex
%
\pdfbookmark[0]{Abstract}{Abstract}
\chapter*{Abstract}\label{sec:abstract}
\vspace*{-10mm}

Given how technology surrounds our whole life, learning to code is becoming more and more crucial for the general public: think for example of the amount of software involved in managing a flight, or when you just turn on the engine of your car. People want to play an increasingly active role in their life and there is already evidence in an overall heightened interest in coding from the many successful public initiatives aiming at introducing coding skills to a wide audience. Nonetheless, coding skills are not just about programming but require an ability of problem-solving, abstraction, pattern recognition to name but a few; in a word, the so-called \ac{CT} skills, namely a set of thinking skills, habits, and approaches that are integral to solving complex problems using a computer and widely applicable in today's information society.

Due to this sudden global interest in promoting \ac{CT} skills to many broad and diverse audiences, several tools and methods have been designed with the aim of supporting the introduction of programming concepts in more effective and less daunting ways than the past. A popular theory of learning that can come to the aid on this matter is Piaget's constructivism, which argues that people produce knowledge and form new meanings based upon their experiences in the real world and social interactions. Thus, exploiting human's natural ability for objects manipulation in the physical world and its afforded interactions could be an effective way of supporting users in learning abstract concepts such as the ones underpinning \ac{CT}.

\aclp{TUI} are an interaction paradigm that was devised to foster collaborative learning and exploit humans' natural dexterity for physical objects manipulation to provide an easy to use interface that can be used even by inexperienced people. They exploit the physical world to offer a concrete representation of the abstract concepts learners usually struggle with and thus employing them to teach those concepts underpinning \ac{CT} might represent an effective and engaging way of supporting the learning of such skills.

This thesis investigates this claim through the development of a software platform combining its digital and physical features to promote \ac{CT} skills in different domains. The platform design is informed by a review of related work, a workshop with domain experts, and was validated through a series of studies in different application scenarios which reported promising results in terms of \ac{CT} support.